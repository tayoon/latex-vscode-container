\documentclass[English]{dicomopapers}

\usepackage[dvips]{graphicx}
\usepackage{latexsym}

\def\Underline{\setbox0\hbox\bgroup\let\\\endUnderline}
\def\endUnderline{\vphantom{y}\egroup\smash{\underline{\box0}}\\}
\def\|{\verb|}

\begin{document}

\title{DICOMO2022 Paper Format}

\affiliate{IPSJ}{Information Processing Society of Japan}
\paffiliate{DICOMO}{DICOMO2022}


\author{TARO JOHO}{IPSJ}
\author{HANAKO SHORI}{DICOMO}

\begin{abstract}
This document is a guide to produce the final camera-ready manuscript
of a paper to be submitted to DICOMO2022, using Japanese {\LaTeX} and
special style files. Since the document itself is produced with the
style files, it will help you to refer its source file which is
distributed with the style files. These style files are based on the
style files for IPSJ Journal and Transactions (available on
https://www.ipsj.or.jp/english/jip/submit/style.html ), so please refer
them to get more information about {\LaTeX} commands, etc.
{\bf \underline{In addition, please use not the format indicated to the
  above}}
{\bf \underline{mentioned manuscript writing guidance but this format. }}
\end{abstract}

\maketitle

%1
\section{Paper Formatting}
There is no restriction on the number of pages. Submitted papers
with full-paper style may be promoted for IPSJ journal.
You can see IPSJ Journal style guide in
the following URL\cite{webpage}.

\noindent
https://www.ipsj.or.jp/english/jip/submit/ronbun\_e\_prms.html

Based on the above guideline, this format is customized for DICOMO2022
to suppress output of IPSJ-style header/footer and keep integrity with
MS Word based format.
These files are redistributed under permission of IPSJ.

For DICOMO 2022, please be aware of the followings.
\begin{itemize}
 \item Style files are customized.
       \begin{itemize}%{
	\item[]\tt dicomopapers.cls
       \end{itemize}%}
 \item Please specify documentclass as\\
      \|\documentclass[english]{dicomopapers}|
 \item Please do not make biography section.
 \item Please do not write in the surrounding margins (header and footer), such as the society name, copyright, page number, etc.
 \item Please do not write in the write an author introduction on the last page.
\end{itemize}
%}

To reduce our effort to make proceedings,
{\bf \underline{please read}}\\
{\bf \underline{style guide carefully. Please don't ask IPSJ for}}\\
{\bf \underline{help on this style file. DICOMO2022 committee}}\\
{\bf \underline{also cannot help on your environment in most}}\\
{\bf \underline{cases.}}

\begin{thebibliography}{99}
\bibitem{webpage}
Information Processing Society of Japan (IPSJ): Journal of Information
Processing (JIP): Information for Authors, IPSJ(online)
\urle{https://www.ipsj.or.jp/english/\\jip/submit/ronbun\_e\_prms.html}
\refdatee{2022-03-01}.
\end{thebibliography}

\end{document}
